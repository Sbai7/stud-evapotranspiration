\begin{abstract}

Die Evapotranspiration ist ein wichtiger Parameter bei der Untersuchung des Wasserkreislaufes, der Modellierung von Einzugsgebieten, der Optimierungen in der Landwirtschaft und in vielen anderen Bereichen. Sie kann entweder direkt aus Lysimeterdaten berechnet oder anhand von empirischen Modellen und meteorologischen Daten modelliert werden. In diesem Versuch werden die empirischen Methoden von Penman-Monteith, Turc und Ivanov verwendet. Alle Daten werden mit MATLAB prozessiert und dargestellt.

In einem ersten Teil wird die Korrelation zwischen der realen Evapotranspiration und den meteorologischen Grössen untersucht. Dabei ist zu erkennen, dass besonders die Globalstrahlung einen entscheidenden Einfluss auf die Evapotranspirationsrate hat und die Windgeschwindigkeit in der Schweiz vernachlässigbar ist.

In einem zweiten Teil werden die verschiedenen Methoden zur Bestimmung der potentiellen Evapotranspiration miteinander verglichen. Die modellierten Evapotranspirationsraten liegen bei allen Modellen in derselben Grössenordnung, wobei in der täglichen Auflösung die Methode nach Turc und Ivanov eine grössere Abhängigkeit von den meteorologischen Parametern Globalstrahlung, Temperatur und relative Luftfeuchtigkeit zeigen als dies mit der Penman-Monteith Methode der Fall ist.

Mit einem Pflanzenfaktor kann die potentielle Evapotranspiration für jede spezifische Pflanzenart bestimmt werden. In diesem Versuch werden die potentiellen Evapotranspirationsraten von Raps und Weizen miteinander verglichen. Diese verhalten sich im jährlichen Verlauf sehr ähnlich, ausser dass die maximale reale Evapotranspiration des Weizens ein Monat nach jener von Raps erreicht wird.

In einem weiteren Teil wird für jede Methode eine Sensitivitätsanalyse durchgeführt. Dabei zeigen bei der Methode nach Penman-Monteith und Ivanov alle Parameter eine etwa gleich grosse Sensitivität. Bei der Turc-Methode ist die relative Luftfeuchtigkeit der sensitivste Faktor.

\end{abstract}
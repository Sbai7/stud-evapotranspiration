\section{Diskussion}




\subsection{Sensitivitätsanalyse}
Die Sensitivitätsanalyse der Penman-Monteith Methode zeigt, dass der Sättigungsdampfdruck den grössten Einfluss auf die Evapotranspiration hat und die Grösse Bodentemperatur invers auf das Resultat Einfluss nimmt. Die Analyse zeigt auch, dass Niederschlag, relative Luftfeuchtigkeit und Sonnenscheindauer vernachlässigbar sind. Dies ist allerdings fragwürdig, da der Niederschlag der limitierende Faktor der Evapotranspiration ist. <<

Bei der Methode nach Turc ist bereits aus der Formel (\ref{eq:turc}) ersichtlich, dass die relative Luftfeuchtigkeit linear in die Berechnung der Evapotranspiration eingeht und somit auch den Grössten Einfluss auf das Resultat hat. Die anderen Grössen fliessen nicht linear ein, was auch die Sensitivitätsanalyse widerspiegelt.

Die Methode nacht Ivanov zeigt bei tiefen Temperaturen eine ausgeglichene Sensitivität auf die beiden Parameter. Für höhere Temperaturen reagiert die Evapotranspiration allerdings sehr viel sensitiver auf die Lufttemperaturen. 
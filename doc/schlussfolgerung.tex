\section{Schlussfolgerung}

Der Versuch hat gezeigt, dass die Evapotranspiration ein sehr komplexer Vorgang ist. Viele verschiedene Umweltfaktoren beeinflussen die Evapotranspirationsrate. Diese Faktoren lassen sich oft nicht einfach bestimmen und ihr Einfluss ist nicht immer direkt ersichtlich. Um die Evapotranspiration modellieren zu können, wurden viele verschiedene empirische Modelle entwickelt und diese sind meist nur für eine klimatische Region gültig. Will man die Evapotranspirationsrate modellieren ist es wichtig, dass man jene Methode auswählt, welche am besten auf die Fragestellung und die geografischen Gegebenheiten passt. So werden zum Beispiel bei der Penman-Monteith Methode zahlreiche meteorologische Grössen berücksichtigt, allerdings wird durch diese Komplexität die direkte Abhängigkeit von den meteorologischen Bedingungen abgeschwächt.
In diesem Versuch wurde ein über das ganze Jahr hinweg konstanter Pflanzenfaktor verwendet. Dieser kann allerdings der Wachstumsphase der jeweiligen Pflanze angepasst werden. Speziell für die Landwirtschaft in ariden Gebieten ist es nützlich, über die Evapotranspiration der Pflanzen Bescheid zu wissen. So können Aussaat, Ernte und Bewässerung dem Klima angepasst werden und die Ressourcen und den Ertrag optimiert werden. 
